\documentclass[letterpaper, 12pt]{article}
\usepackage[top=1in, left=1in, right=1in, bottom=1in]{geometry}
\usepackage[utf8]{inputenc}
\usepackage{enumitem}
\usepackage{booktabs}
\usepackage{fancyhdr}
\usepackage{amsmath}
\usepackage{hyperref}
\usepackage{listings}
\lstset{language=bash, basicstyle=\ttfamily}

\title{A Software Introduction to the Concordia Aircraft Systems Research Lab}
\author{}
\date{}

%\pagestyle{fancy}
%\lhead{}
%\rhead{}

\begin{document}
\maketitle
\tableofcontents
\section{Introduction}
Welcome to the Concordia Aircraft Systems Research Lab! This document will serve as an introduction to some of the software tools we use, as well as some programs and links to get you started if you're new to working with Python and software development.

\section{Python}
Python is one of the most popular programming languages today, due to it having simple syntax and being easy to learn, having a large community which develops useful libraries and makes it easy to find help, and it being free. 

\subsection{Where to find help}
Given Python's popularity, there is a lot of help available online. The Python Foundation lists many tutorials \href{https://wiki.python.org/moin/BeginnersGuide/Programmers}{on this page}. They also have created their own tutorial \href{https://docs.python.org/3/tutorial/}{here}, and if you have the time for it, MIT OpenCourseWare has provided a full free introductory course on Python \href{https://ocw.mit.edu/courses/electrical-engineering-and-computer-science/6-0001-introduction-to-computer-science-and-programming-in-python-fall-2016/lecture-videos/}{here}.

For more specific questions you might have about Python, Googling for it will usually lead to finding an answer through Stack Overflow posts or programming websites.

\subsection{Creating and Running Python Files}
Python files are text files with the extension \emph{.py}. When files have this extension, programs (such as the text editor you are using to edit the Python file) will recognize the file as being a Python file and might enable some Python-specific formatting features. It will also allow you to run these files through the Python interpreter. 

To run Python files, you may want to use Conda, which is covered in Section \ref{anaconda}


\subsection{Python Style Guide}
Part of what makes Python so appealing is that it is pretty and readeable. While some of that is due to Python's inherent structure (for example, you must tab to be inside of a loop of function), it is mostly due to the style conventions outlined in \href{https://www.python.org/dev/peps/pep-0008/}{PEP-8} (PEP = Python Enhancement Proposal). This document describes style conventions which are meant to make Python be as readeable as possible, as code is read more often than it is written. It is highly recommended to give it a read, as it is not too long, and will help you write beautiful code.

Given the importance of readeability and following the style guide (also known as code linting), there are many tools to help enforce it. Most modern text editors will recognize that you are writing a Python file (thanks to the file extension) and set the tab width to be equal to 4 spaces, as defined in PEP-8. They might also auto-tab for you once you define a function or create a loop. However to make sure that you are respecting the style guide, you can use the \href{https://flake8.pycqa.org/en/latest/}{flake8} package. By running this package on your Python file, your code will be checked against PEP-8 and have every style infraction output to the console. You can fix those errors and rerun flake8 until you do not get any more errors.

\subsection{Documentation}
Documenting your code and projects are important, as people are not likely to read through your actual code to figure out what it does and if they should use it. It also helps you understand your code, as over long-term projects you might not remember how the code you wrote at the beginning of the project works. 

\href{https://www.python.org/dev/peps/pep-0257/}{PEP-257} describes how you should document your functions and classes using Document Strings, or docstrings for short. Using this format can also let you automatically generate documentation pages using the \href{https://docs.python.org/3/library/pydoc.html}{pydoc} module. You should also use comments within the code, PEP-8 describes the style you should use, and \href{https://python-102.readthedocs.io/en/latest/documenting.html}{this tutorial} gives examples for how to comment.

When creating a repository in GitHub (which will be covered in more detail in Section \ref{github}), you will want to create a README.md file to describe the project in that repository. Github has created \href{https://help.github.com/en/github/creating-cloning-and-archiving-repositories/about-readmes}{this page} describing what typically goes in a README.md file. These files use the Markdown formatting language, and \href{https://help.github.com/en/github/writing-on-github/basic-writing-and-formatting-syntax}{this page} describes how to use it.

\section{Useful Programs}
This section describes some tools that aid in developing tools and writing scripts.
\subsection{GitHub}\label{github}
\href{https://github.com/}{GitHub} is a version control system that uses the Git program to save your projects in the cloud and allow for collaboration with others. GitHub (and other Git-based version control hosting platforms) are ubiquitous in software development as it makes it easy to both track changes made to programs and collaborate on the same code without hassle.

To get started, \href{https://github.com/join}{create a GitHub account}. Make sure to choose the Free plan. Once your account is made, you can transform your account into an educational account, which gives you all the perks of the Pro plan for free. To do so, \href{https://help.github.com/en/github/teaching-and-learning-with-github-education/applying-for-a-student-developer-pack}{follow these instructions}.

A great Git (and GitHub) tutorial is \href{https://guides.github.com/activities/hello-world/}{this Hello World tutorial}, which will introduce you to the Git workflow. If you'd like to learn more about some of the more advanced GitHub commands not covered in the Hello World tutorial, \href{https://guides.github.com/introduction/git-handbook/}{this page} contains some more definitions and has links to other tutorials, projects, and videos.

As a lab, all shared tools are created in the \href{https://github.com/ConcordiaAircraftSystemsLab}{ConcordiaAircraftSystemsLab} GitHub Organization. Organizations are spaces where multiple people can collaborate and access repositories. The advantage is that all the Organization's tools are in the same spot, rather than having different tools existing in the repositories of various members. To gain access to the Lab's GitHub Organization, you will need an admin (as of time of writing, this is Prof. Liscou{\"e}t-Hanke) to add you to it using your GitHub username.

To use GitHub without needing to go through the browser interface, the \href{https://desktop.github.com/}{GitHub Desktop} application is a GUI tool that makes it easy to manage your repositories and interface between your computer and GitHub. It is recommended to download and use this tool.

For command-line usage of GitHub, Windows users can install the \href{https://gitforwindows.org/}{Git for Windows console}. Linux and MacOS users can install Git through their package manager and use their native Terminal program.

As a summary, GitHub is useful for us as it allows for collaboration on tools, keeping track of changes to code, and for creating a single space that we all have access to for working on those tools.

\subsection{Text Editors}
Python files can be written using any text editor, though some will have nice code highlighting and some automatic formatting which make it much easier to read and write code. One popular editor is \href{https://code.visualstudio.com/}{VSCode}, which has these features, and also supports GitHub so you can see which lines in a given file have been changed since the last commit. Documentation about using VSCode with Python (and other programming languages) is found \href{https://code.visualstudio.com/docs/languages/python}{here}, and you can install a VSCode extension for Python \href{https://marketplace.visualstudio.com/items?itemName=ms-python.python}{here}.

For an excellent text editor without the bells and whistles, download \href{https://notepad-plus-plus.org/}{Notepad++}. While VSCode is a much more pleasant environment to program in, Notepad++ opens files super quickly and lets you view and edit files without having to open up VSCode, which can take a few seconds to open. It is also useful for opening and reading/writing to any text file, including CSV files.

\subsection{Conda}\label{anaconda}
\href{https://docs.conda.io/projects/conda/en/latest/index.html}{Conda} is a package manager for programming languages, and has gained a lot of popularity for usage with Python. A package manager is a program that lets you easily download packages (such as installing numPy) from one or several package repositories. What makes a package manager special is that it will ensure that if you install a package, all its dependencies (other packages that it depends on to run properly) are also installed. It will also keep track of any updates to packages, so with one command (\lstinline{conda update}) you can update all your packages to the most recent versions.

However, what makes Conda special is that it is also an environment manager. An environment manager lets you create environments, and install packages within that environment that do not affect other environments. For example, you might have one environment where you have Python 3.6 installed, as well as a whole host of packages that are compatible with that version of Python. If you also want to have Python 2.7 installed for running some legacy application, you can create a new environment with that version of Python and the packages that work with it. In this way, you now have two completely separate installations of Python and assorted packages that all work well together within their environment, but cannot impact each other outside of it. For more information about environments, \href{https://docs.conda.io/projects/conda/en/latest/user-guide/concepts/environments.html}{view this document}

To use Conda, you must \href{https://docs.anaconda.com/anaconda/install/}{install Anaconda}, which is a collection of tools that Conda is a part of. It is recommended to go through the \href{https://docs.conda.io/projects/conda/en/latest/user-guide/getting-started.html}{Conda tutorial}, which will explain how to install and update packages, create and use environments, and switch between environments.

Conda in general is useful for us as, by default, it comes with most of the Python packages we would typically use for development, and makes it very easy to install new packages that we need. Environments are useful as we can develop tools using specific versions of packages, and as long as other users of those tools have the same environment, it is certain that the tools will work for them. It also means that we can install programs and use environements that the program developers define to ensure that those programs will work for us.

\subsection{Jupyter Lab}
\href{https://jupyterlab.readthedocs.io/en/stable/getting_started/overview.html}{Jupyter Lab} is a browser-based program that lets you create and edit Jupyter Notebooks, which are documents which contain live code, text, graphs, equations, and more! Notebooks are made up of \lq\lq cells\rq\rq, each of which can contain code and have their output expressed below. This format is conducive to exploring or documenting a workflow. For some relevant examples, look at the notebooks available at \href{https://hub.gke.mybinder.org/user/sobester-adrpy-010agxkw/notebooks/docs/ADRpy/notebooks/Initial%20scaling%20of%20an%20aircraft%20with%20ADRpy%20-%20the%20climb%20constraint.ipynb}{this link} (Notebooks have the \emph{.ipynb} extension), which come from \href{https://adrpy.readthedocs.io/en/latest/}{this program}. \href{https://github.com/jupyter/jupyter/wiki/A-gallery-of-interesting-Jupyter-Notebooks}{This GitHub link} documents scores of example Jupyter Notebooks and tutorials for a variety of different programming applications.

Since Jupyter Lab comes with Anaconda, your installation of Anaconda should have also installed Jupyter Lab on your computer. If for whatever reason it has not, you can follow the installation instructions for Jupyter Lab \href{https://jupyterlab.readthedocs.io/en/stable/getting_started/installation.html}{here}. For usage, \href{https://jupyterlab.readthedocs.io/en/stable/getting_started/overview.html}{this link} gives a good overview of Jupyter Lab and \href{https://jupyterlab.readthedocs.io/en/stable/user/interface.html}{this link} gives a user guide. \href{https://ipython-books.github.io/36-introducing-jupyterlab/}{This tutorial} is concise and explains how to use Jupyter Lab and notebooks, as well as covering the differences between Jupyter Lab and the Jupyter Notebooks App.

\section{An Example Workflow}
This section goes through a typical/potential development workflow, starting from creating some code, running that code, creating a GitHub repository and adding your files to it, and sharing that code with others for use and further development.

\subsection{Python File Creation and Editing}
To start with, decide whether you are starting a new project or writing some miscellaneous short script. If starting a new project, it is best to create a new directory (or folder) for it, but if not you can put your new python file wherever you find convenient. Next, you can create your Python file. One way to do so is to create an empty text file with the \emph{.py} file extension and opening that in VSCode. You can also start up VSCode, clicking \lq\lq open folder...\rq\rq, navigating to your chosen folder and either clicking \lq\lq New File\rq\rq or right-clicking in the VSCode file explorer and creating a new file that way. If you created a file and want to open it using VSCode, navigate to the folder that file is in and double click the file name in the VSCode file explorer to have it open up.

At this point, you can start writing some code in the file. You can save with CTRL-S or by clicking File $\rightarrow$ Save.

\subsection{Running the Code}
To run the code, we will use Python in Anaconda. If using Windows, you will want to open Anaconda Prompt, which is a program that you can search for in the Start menu and is included with your Anaconda installation. For Linux or Mac, open a Terminal window and type \lstinline{conda activate}. Once done, navigate to the folder that contains your Python file through the Anaconda Prompt or Terminal window. To navigate to a directory, you can use the \lstinline{cd} command, where you would type a command like \lstinline{cd path/to/folder}. In Windows you can type \lstinline{dir} to display a list of all the files in that folder.

Next, to run your Python file, type \lstinline{python filename.py}, where filename.py is the file you want to run. 

\subsection{Creating a GitHub Repository}
Now that you have some useful code that you want to share or have others contribute to, you may want to add it to GitHub. This is done by creating a GitHub repository. If what you are working on is a tool for the Lab, you will want to create it in the ConcordiaAircraftSystemsLab Organization. If not, you can create it on your personal page.

To create a repository using the GitHub Desktop application, you can follow \href{https://help.github.com/en/desktop/getting-started-with-github-desktop/creating-your-first-repository-using-github-desktop}{these instructions}. You can also create a repository using GitHub through your browser by following \href{https://help.github.com/en/github/getting-started-with-github/create-a-repo}{these instructions}

If you created the repository using GitHub Desktop, you will have already set the top-level folder on your computer that corresponds to that repository, so your Python file (and any other files in that folder) should now be part of that repository.

If you created the repository using GitHub through the browser, you will have to add your folder to it using \href{https://help.github.com/en/github/managing-files-in-a-repository/adding-a-file-to-a-repository}{these instructions}. Note that you will need to re-upload files to the repository every time you want to make a commit, so this can be time-consuming and ineffective way to work with GitHub. A better way would be to upload your files, then delete them from your computer and use GitHub Desktop to \lq\lq clone\rq\rq the GitHub repository to your computer by following \href{https://help.github.com/en/desktop/contributing-to-projects/cloning-and-forking-repositories-from-github-desktop}{these instructions}.

You can also use the command line (for Windows you will need to install \href{https://gitforwindows.org/}{the Git Bash application}) by following \href{https://help.github.com/en/github/importing-your-projects-to-github/adding-an-existing-project-to-github-using-the-command-line}{these instructions}.

\subsection{Sharing the GitHub Repository}
Now you can share the GitHub repository. If it is a public repository, which you would have defined when you created the repository (and you can always convert between private and public in the future), you can send someone the link to the repository and they will be able to access it.

To invite a GitHub user to be able to push commits to your repository, you can add them as a Collaborator by going to the repository page, going to \lq\lq Settings\rq\rq, then \lq\lq Manage access\rq\rq, then \lq\lq invite teams or people\rq\rq.

\end{document}
